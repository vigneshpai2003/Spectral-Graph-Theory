\documentclass{math}

\usepackage{bm}

\renewcommand{\vec}[1]{\boldsymbol{#1}}

\title{Spectral Graph Theory}
\author{Vignesh M Pai}
\date{}


\begin{document}

\maketitle

\tableofcontents

\section{Introduction}

\subsection{Eigenvalues and Optimization}

Let $M$ be a $n$ dimensional symmetric matrix.

\begin{definition}[Rayleigh Quotient]
    The Rayleigh quotient for a matrix $M$ is defined as
    \begin{align*}
        R(\vec{x}, M) := \frac{\vec{x}^T M \vec{x}}{\vec{x}^T \vec{x}}
    \end{align*}
    the matrix is ommitted if obvious from context.
\end{definition}

\begin{theorem}
    Let
    \begin{align*}
        \vec{x} \in \arg \max_{\vec{x} \in \mathbb{R}^n - \set{0}} R(\vec{x})
    \end{align*}
    Such an $\vec{x}$ exists and is an eigenvector of $M$ with the maximum eigenvalue $\mu_1$.
    We can write a similar statement for the minimum eigenvalue by minimizing $R$.
\end{theorem}

\begin{theorem}[Spectral Theorem for Symmetric Matrices]
    There exist numbers $\mu_1, ..., \mu_n$ and orthonormal vectors $\vec{\psi}_1, ..., \vec{\psi}_n$ such that $M \vec{\psi}_i = \mu_i \vec{\psi}_i$
    iff for $1 \leq i \leq n$
    \begin{align*}
        \vec{\psi}_i \in \arg \max_{\substack{\norm{\vec{x}} = 1 \\ \vec{x}^T \vec{\psi}_j = 0,\ j < i}} R(\vec{x})
    \end{align*}
    or equivalently
    \begin{align*}
        \vec{\psi}_i \in \arg \min_{\substack{\norm{\vec{x}} = 1 \\ \vec{x}^T \vec{\psi}_j = 0,\ j > i}} R(\vec{x})
    \end{align*}
\end{theorem}

\begin{theorem}[Courant-Fischer Theorem]
    Let $M$ have eigenvalues $\mu_1 \geq \mu_2 \geq ... \geq \mu_n$, then
    \begin{align*}
        \mu_k = \max_{\substack{S \subset \mathbb{R}^n \\ \dim(S) = k}} \min_{\substack{\vec{x} \in S \\ \vec{x} \neq 0}} R(\vec{x}) = \min_{\substack{T \subset \mathbb{R}^n \\ \dim(T) = n - k + 1}} \max_{\substack{\vec{x} \in T \\ \vec{x} \neq 0}} R(\vec{x})
    \end{align*}
    where $S, T$ are subspaces of $\mathbb{R}^n$.
\end{theorem}

\begin{theorem}[Cauchy's Interlacing Theorem]
    Let $A$ be a symmetric real matrix of dimension $n$.
    Let $B$ be obtained by deleting the same row and column of $A$ ($N$ is a principal submatrix of dimension $n - 1$).
    Let $\alpha_1 \geq ... \geq \alpha_n$ be the eigenvalues of $A$ and $\beta_1 \geq ... \geq \beta_{n-1}$ be the eigenvalues of $B$.
    Then for $1 \leq k \leq n - 1$
    \begin{align*}
        \alpha_k \geq \beta_k \geq \alpha_{k + 1}
    \end{align*}
\end{theorem}

\subsection{The Laplacian and Graph Drawing}

Vectors on graphs are functions $V \to \mathbb{R}$, the vector $\vec{1}$ denotes the function $\vec{1}(a) = 1$.
The degree function $\vec{d}$ is also a vector.

Matrices on graphs are functions $V \times V \to \mathbb{R}$ or can be viewed as linear operators on the space of vectors on graphs.
Let $G$ be a graph, then we use $M_G$ to denote the adjacency matrix,
$D_G$ to denote the diagonal matrix of vertex degrees,
and $L_G = D_G - M_G$ to denote the Laplacian matrix.
Observe that $M_G \vec{1} = \vec{d}$ and $L_G \vec{1} = \vec{0}$.

The Laplacian is also a natural quadratic form on a graph
\begin{align*}
    \vec{x}^T L_G \vec{x} = \sum_{(a, b) \in E(G)} w_{a, b} (\vec{x}(a) - \vec{x}(b))^2
\end{align*}
Let $\lambda_1 = 0 \leq \lambda_2 \leq ... \leq \lambda_n$ be the eigenvalues of $L_G$ with eigenvectors $\vec{\psi}_1, ..., \vec{\psi}_n$.

\begin{lemma}
    $G$ is connected iff $\lambda_2 \neq 0$.
\end{lemma}

We draw a graph in $k$ dimensions by using the eigenvectors corresponding to $\lambda_2, ..., \lambda_{k+1}$ as the coordinates of vertices.
These coordinates minimize the expression (excluding coordinates that lead to trivial drawings):
\begin{align*}
    \sum_{(a, b) \in G} w_{a, b} \norm{x(a) - x(b)}^2 = \sum_{i=1}^{k} \vec{x_i}^T L_G \vec{x_i}
\end{align*}
where $x: V \to \mathbb{R}^k, x = (\vec{x_1}, ..., \vec{x_k})$ is the coordinate function.

\begin{theorem}[Hall's Drawing Theorem]
    Let $\vec{x_1}, ..., \vec{x_k}$ be orthonormal vectors that are orthogonal to $\vec{1}$, then
    \begin{align*}
        \sum_{i=1}^{k} \vec{x_i}^T L_G \vec{x_i} \geq \sum_{i=2}^{k + 1} \lambda_1
    \end{align*}
    where equality holds for $\vec{\psi}_j^T \vec{x_i} = 0$ for all $i$ and $j > k + 1$.
\end{theorem}

\subsection{Adjacency Matrices}

Let the eigenvalues of $M_G$ be $\mu_1 \geq ... \geq \mu_n$.

\begin{lemma}
    Let $d_{avg}$ and $d_{max}$ be the average and maximum degrees respectively, then
    \begin{align*}
        d_{avg} \leq \mu_1 \leq d_{max}
    \end{align*}
    further, if $H$ be a subgraph of $G$, then
    \begin{align*}
        d_{avg}(H) \leq \mu_1
    \end{align*}
\end{lemma}

\begin{lemma}
    If $G$ is connected and $\mu_1 = d_{max}$ then $G$ is $d_{max}$-regular.
\end{lemma}

\begin{theorem}[Wilf's Theorem]
    Let $\chi(G)$ be the chromatic number of $G$, then
    \begin{align*}
        \chi(G) \leq \lfloor\mu_1\rfloor + 1
    \end{align*}
\end{theorem}

\begin{lemma}
    Let $G$ be a connected weighted graph and $\vec{\phi}$ be a non negative eigenvector of $M_G$, then $\vec{\phi}$ is strictly positive.
\end{lemma}

\begin{theorem}[Perron-Frobenius Theorem]
    Let $G$ be connected, then
    \begin{enumerate}
        \item $\mu_1$ has a strictly positive eigenvector
        \item $\mu_1 \geq - \mu_n$
        \item $\mu_1 > \mu_2$
    \end{enumerate}
\end{theorem}

\begin{lemma}
    If $G$ is bipartite, then the eigenvalues of $M_G$ are symmetric about $0$.
\end{lemma}

\begin{theorem}
    Let $G$ be connected, $\mu_1 = - \mu_n$ iff $G$ is bipartite.
\end{theorem}

\subsection{Comparing Graphs}

We introduce the partial order $\succeq$ on matrices as
\begin{align*}
    A \succeq B \iff \forall \vec{x}, \vec{x}^T A \vec{x} \geq \vec{x}^T B \vec{x}
\end{align*}
In particular $A \succeq 0$ means $A$ is positive semidefinite.
For graphs $G, H$ on the same set of vertices, we write $G \succeq H$ iff $L_G \succeq L_H$.
If $H$ is a subset of $G$, we have
\begin{align*}
    G \succeq H
\end{align*}

For a graph $H$, define $c \cdot H$ to be the graph $H$ with each edge weight multiplied by $c$.
Let $\lambda_k(H)$ denote the $k$th smallest eigenvalue of $L_H$.
\begin{lemma}
    If $G, H$ are graphs
    \begin{align*}
        G \succeq c \cdot H \implies \lambda_k(G) \geq c \lambda_k(H)
    \end{align*}
\end{lemma}

Let $G_{a, b}$ be the graph with only the edge $(a, b)$.
The proof of the following lemmas follow trivially from the Cauchy Schwarz inequality applied to the Laplacian.

\begin{lemma}
    Let $P_n$ be the path graph on $n$ vertices between vertex $1$ and $n$.
    \begin{align*}
        G_{1, n} \preceq (n - 1)P_n 
    \end{align*}
\end{lemma}

\begin{lemma}[Extension to Weighted Paths]
    Let $P_{n, w}$ be the weighted path graph on $n$ vertices with $w_i$ the weight on the edge $(i, i+1)$.
    \begin{align*}
        G_{1, n} \preceq \left(\sum_{i=1}^{n-1} \frac{1}{w_i}\right) P_{n, w}
    \end{align*}
\end{lemma}

We use these lemmas to prove bounds on the eigenvalues of graphs.

Let $K_n$ be the complete graph on $n$ vertices, it is easy to see that $\lambda_i(K_n) = n$ for $i \geq 2$.
We can also write that
\begin{align*}
    L_{K_n} = \sum_{a < b} L_{G_{a, b}}
\end{align*}
this allows us to prove the following
\begin{gather*}
    K_n = \sum_{a < b} G_{a, b} \preceq \sum_{a < b} (b - a) P_{a, b} \preceq \sum_{a < b} (b - a) P_n \\
    \implies \lambda_2(K_n) = n \leq \lambda_2(P_n) \sum_{a < b} (b - a) = \frac{n (n + 1) (n - 1)}{6} \lambda_2(P_n)
\end{gather*}

\begin{lemma}[Bounding $\lambda_2$ of the Path Graph]
    \begin{align*}
        \lambda_2(P_n) \geq \frac{6}{(n+1)(n-1)}
    \end{align*}
\end{lemma}

Let $T_d$ be the complete binary tree of depth $d$, this will have $n = 2^{d+1} - 1$ vertices.
Let $T_d^{a, b}$ be the shortest path between vertices $a, b$ on $T_d$.
Note that this path has size atmost $2d \leq 2 \log_{2} n$. Doing a similar comparision with $K_n$ we get
\begin{gather*}
    K_n = \sum_{a < b} G_{a, b} \preceq \sum_{a < b} 2d T_d^{a, b} \preceq \sum_{a < b} 2 \log_2 n T_d = \binom{n}{2} 2 \log_2 n T_d \\
    \implies \lambda_2(K_n) = n \leq \binom{n}{2} 2 \log_2 n \lambda_2(T_d)
\end{gather*}

\begin{lemma}[Bounding $\lambda_2$ of Complete Binary Trees]
    \begin{align*}
        \lambda_2(T_d) \geq \frac{1}{(n - 1) \log_2 n}        
    \end{align*}
\end{lemma}

\section{Random Graphs}

An Erd\"{o}s-R\'{e}nyi random graph is a graph in which each edge is present with probability $p$ independent of other edges.
We can write the adjacency matrix $M$ of this graph as
\begin{align*}
    M = p(J - I) + R
\end{align*}
where $J$ is the all ones matrix, $I$ is the identity matrix and $R$ is defined for off diagonal entries as
\begin{align*}
    R(a, b) = \begin{cases}
        1 - p & \text{with probability $p$} \\
        -p & \text{with probability $1-p$}
    \end{cases}
\end{align*}
clearly the expectation of $M$ is $\rho(J - I)$, which means the expectation of $R$ is the zero matrix.
This can be easily verified from the definition of $R$.

\end{document}
